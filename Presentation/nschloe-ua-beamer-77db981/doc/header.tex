\documentclass[%
paper=a4,
parskip=half,
oneside
]{scrartcl}

\usepackage[T1]{fontenc}
\usepackage{lmodern}
\usepackage{microtype}

% For the Auto 1 example at the end.
\usepackage{fontspec}
\defaultfontfeatures{Ligatures=TeX, Scale=MatchLowercase}

\usepackage{listings}  % include highlighted source code
\lstset{% general command to set parameter(s)
basicstyle=\ttfamily, % print whole listing small
keywordstyle=\color{uablue},
identifierstyle=,
commentstyle=\color{olive},
stringstyle=\color{uared},
showstringspaces=false,     % no special string spaces
frame=single,
language=[LaTeX]TeX
}

\usepackage[amsthm]{ntheorem}
\theoremstyle{definition}
\newtheorem*{note}{Note}

\usepackage{booktabs}

\usepackage{xcolor}

\usepackage{graphicx}

\usepackage{tabulary}

\usepackage{subfig}

\usepackage{csquotes}

\usepackage{eurosym}

\usepackage{tikz}
\usepackage{pgfplots}

% \usepackage[numbers]{natbib}
% \bibliographystyle{plainnat}
\usepackage[%
%bibtex8=true,
hyperref=auto
]{biblatex}
\bibliography{bibtex/beamerstyleUa}

\newcommand{\changefont}[3]{\fontfamily{#1}\fontseries{#2}\fontshape{#3}\selectfont}

\usepackage[
bookmarks,
colorlinks,
linkcolor=uared,
citecolor=vividbrown,
urlcolor=uared75,
pdftitle={A beamer presentation class theme for the Universiteit Antwerpen},
pdfauthor={Nico Schloemer}
]{hyperref}


\newlength{\figurewidth}
\newlength{\figureheight}

% =========================================================================
\title{A \texttt{beamer} presentation class theme for the Universiteit Antwerpen}
\author{Nico~Schl\"omer\thanks{e-mail: \href{mailto:nico.schloemer@gmail.com}{\nolinkurl{nico.schloemer@gmail.com}}. Comments and suggestions always welcome.}}
\date{Version 0.2, \today}
% =========================================================================


% =========================================================================
% define colors
% =========================================================================
% see PMS-CMYK-RGB conversion chart at http://www.zedimage.com/pms-cmyk-hex.php
\definecolor{uablue}{cmyk}{1,0.25,0,0.5}
\colorlet{uablue100}{uablue}
\colorlet{uablue75} {uablue!75!white}
\colorlet{uablue50} {uablue!50!white}
\colorlet{uablue25} {uablue!25!white}
\colorlet{uablue10} {uablue!10!white}
\colorlet{uablue5}  {uablue!5!white}

\definecolor{uared}{cmyk}{0,1,0.6,0.37}
\colorlet{uared100}{uared}
\colorlet{uared75} {uared!75!white}
\colorlet{uared50} {uared!50!white}
\colorlet{uared25} {uared!25!white}
\colorlet{uared10} {uared!10!white}
\colorlet{uared5}  {uared!5!white}

\definecolor{vividbrown}{RGB}{215,154,70}
\colorlet{vividbrown100}{vividbrown}
\colorlet{vividbrown75} {vividbrown!75!white}
\colorlet{vividbrown50} {vividbrown!50!white}
\colorlet{vividbrown25} {vividbrown!25!white}
\colorlet{vividbrown10} {vividbrown!10!white}
\colorlet{vividbrown5}  {vividbrown!5!white}
% =========================================================================


\newcommand\colorsquare[2]{\colorbox{#2}{\rule{0mm}{#1}\rule{#1}{0mm}}}
\newcommand\colorsquareseries[2]{%
\colorsquare{#1}{#2}%
\colorsquare{#1}{#2!75!white}%
\colorsquare{#1}{#2!50!white}%
\colorsquare{#1}{#2!15!white}%
\colorsquare{#1}{#2!10!white}%
\colorsquare{#1}{#2!5!white}%
}

% % =========================================================================
% % A colorwheel as generated by http://www.colorschemer.com/online.html
% % with the input value RGB(0,112,128).
% % Neither of the two UA colors will exactly fit in there, but both approximately
% % at positions (1,1) and (3,4), respectively.
% % Pick the complimentary colors from that wheel.
% \definecolor{col11}{RGB}{  0,111,128}  % approx. uablue
% \definecolor{col12}{RGB}{  0, 47,128}
% \definecolor{col13}{RGB}{ 17,  0,128}
% \definecolor{col14}{RGB}{ 81,  0,128}
% 
% \definecolor{col21}{RGB}{  0,128, 81}
% \definecolor{col22}{RGB}{  0,164,189}
% \definecolor{col23}{RGB}{  0,217,250}
% \definecolor{col24}{RGB}{128,  0,111}
% 
% \definecolor{col31}{RGB}{  0,128, 17}
% \definecolor{col32}{RGB}{250, 33,  0}
% \definecolor{col33}{RGB}{189, 25,  0}
% \definecolor{col34}{RGB}{128,  0, 47} % approx. uared
% 
% \definecolor{col41}{RGB}{ 47,128,  0}
% \definecolor{col42}{RGB}{111,128,  0}
% \definecolor{col43}{RGB}{128, 81,  0}
% \definecolor{col44}{RGB}{128, 17,  0}
% % =========================================================================
